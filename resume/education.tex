%-------------------------------------------------------------------------------
%	SECTION TITLE
%-------------------------------------------------------------------------------
\cvsection{Образование}


%-------------------------------------------------------------------------------
%	CONTENT
%-------------------------------------------------------------------------------
\begin{cventries}

%---------------------------------------------------------
  \cventry
    {Бакалавр, Прикладная математика и информатика} % Degree
    {Московский авиационный институт} % Institution
    {Москва} % Location
    {Сентябрь 2013 - Июль 2016} % Date(s)
    {
      \begin{cvitems} % Description(s) bullet points
        \item {Разработка программного комплекса подготовки полетных заданий группировке беспилотных летательных аппаратов морского базирования. Комплекс включал в себя как расчетную часть, так и визуализацию результатов на фоне электронно-топографической карты, хранение результатов расчета, позволял формировать отчетные документы в формате PostScript. Клиентская часть написана на С++ с использованием Qt, для работы с картографией был использован геодезический API. Была спроектирована и реализована база данных PostgreSQL.}
        %\item {Изучила базовые принципы работы с OpenGL и использование Cuda для параллельных расчетов на матрицах.}
      \end{cvitems}
    }
  \cventry
    {Инженер, Оптико-электронные приборы и системы} % Degree
    {Московский Государственный Университет Приборостроения и Информатики} % Institution
    {Москва} % Location
    {Сентябрь 2000 - Июль 2006} % Date(s)
    {
      \begin{cvitems} % Description(s) bullet points
        \item {Теорема Шеннона: пропускная способность канала. Была разработана лабораторная работа для изучения теоремы кодирования канала с шумом. Реализована визуализация условного процесса передачи информации, ввод исходных данных, хранение результатов работы для каждого студента.}
        %\item {Преподавала у студентов по этой лабораторной работе с использованием реализованной программы.}
      \end{cvitems}
    }

%---------------------------------------------------------
\end{cventries}
